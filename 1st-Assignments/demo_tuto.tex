\documentclass[a4paper,12pt]{article}
\usepackage[utf8]{inputenc}
\usepackage{amsmath}
\usepackage{amsfonts}
\usepackage{amssymb}
\usepackage{graphicx}
\usepackage{hyperref}
\usepackage{fancyhdr}
\usepackage{geometry}
\pagenumbering{gobble}
% Geometry settings
\setlength{\headheight}{14.5pt}
\geometry{
	a4paper,
	left=25mm,
	right=25mm,
	top=20mm,
	bottom=20mm,
}
% Header and footer settings
\pagestyle{fancy}
\fancyhf{}
\fancyhead[L]{\textbf{NITK Surathkal}}
\fancyhead[R]{\textbf{Assignment-$1$}}
\fancyfoot[C]{\thepage}

% Title settings
\title{

		\begin{minipage}{0.15\textwidth}
			\includegraphics[width=\textwidth]{NITK_Emblem.png} % Adjust the size as needed
		\end{minipage}%
		\begin{minipage}{0.85\textwidth}
			\centering
			\textbf{National Institute of Technology Karnataka, Surathkal}
		\end{minipage}
}
\author{
	\begin{minipage}[t]{0.45\textwidth}
		\flushleft{\textbf{Dileep Reddy Gosu}}
	\end{minipage}
	\begin{minipage}[t]{0.45\textwidth}
		\flushright
		\texttt{\textbf{2420585}}
	\end{minipage}
}
\date{
 \begin{minipage}[t]{0.40\textwidth}
	\flushleft
	CSE
\end{minipage}%
\begin{minipage}[t]{0.5\textwidth}
	\flushright
	\today
	\end{minipage}}

\begin{document}
	
	\maketitle
	\headrule
	\vspace{6cm}
	\begin{center}
		\textbf{\Large Course:} \large{M.Tech in Computer Science and Engineering} \\
		\vspace{0.5cm}
		\textbf{\Large Course Code \& Subject:} \large{CS700 \& Algorithm and Complexity} \\
		\vspace{0.5cm}
		\textbf{\Large Professor:} \large{Dr. Vani} \\
		\vspace{0.5cm}
		\textbf{\Large Assignment No.:} \large{1} \\
		\vspace{0.5cm}
		\textbf{\Large Submission Date:} \large{\today} \\
	\end{center}
	\clearpage
	\tableofcontents
	\clearpage
	\pagenumbering{arabic}
	\setcounter{page}{1}
	\section*{Introduction}
	Brief introduction about the assignment.
	\section*{Question 1}
	Detailed explanation or answer to question 1.
	\section{Data Generation and Experimental Setup}
	\begin{enumerate}
		\item \textbf{What kind of machine did you use?}\\ 
		 Acer Nitro5
		\item \textbf{What timing mechanism did you use?} \\
		c++ Standard Timing
		\item \textbf{How many times did you repeat each experiment?}\\
		25 times
		\item \textbf{How did you select inputs?} \\
		Randomly with uniform distribution of positive integers in the range of 1 and 1000000
		\item \textbf{Did you use the same inputs for all sorting algorithms?} \\
		yes
	\end{enumerate}
	\section{Which of the Three Versions of Quicksort Seems to Perform the Best?}
	\begin{enumerate}
		\item \textbf{Graph the best case running time as a function of input size $n$ for all three versions.}
		\begin{figure}[h]
			\centering
			\includegraphics[width=0.75\linewidth]{./pictures/Best_quick.jpg}
			\caption{Best Case for all the Sorting Algorithms.}
		\end{figure}
		\item \textbf{Graph the worst case running time as a function of input size $n$ for all three versions.}
		\begin{figure}[h]
			\centering
			\includegraphics[width=0.75\linewidth]{./pictures/worst_quick.jpg}
			\caption{Worst Case for all the Sorting Algorithms.}
		\end{figure}
		\item \textbf{Graph the average case running time as a function of input size $n$ for all three versions.}
		\begin{figure}[h]
			\centering
			\includegraphics[width=0.75\linewidth]{./pictures/average_quick.jpg}
			\caption{Average Case for all the Sorting Algorithms.}
		\end{figure}
		\begin{figure}[h]
			\centering
			\includegraphics[width=0.75\linewidth]{./pictures/quick_average_expand.jpg}
			\caption{Average Case for all the Sorting Algorithms.Expanded Version}
		\end{figure}
	\end{enumerate}
	\newpage
	\section{Which of the Six Sorts Seems to Perform the Best (Consider the Best Version of Quicksort)?}
	\begin{enumerate}
		\item \textbf{Graph the best case running time as a function of input size $n$ for the six sorts.}
		\begin{figure}[h]
			\centering
			\includegraphics[width=0.75\linewidth]{./pictures/Best.jpg}
			\caption{Best Case for all the Sorting Algorithms.}
		\end{figure}
		\clearpage
		\begin{figure}[h]
			\centering
			\includegraphics[width=0.75\linewidth]{./pictures/Best_enlarged.jpg}
			\caption{Best Case for all the Sorting Algorithms.Expanded Version}
		\end{figure}
		\item \textbf{Graph the worst case running time as a function of input size $n$ for the six sorts.}
		\begin{figure}[h]
			\centering
			\includegraphics[width=0.75\linewidth]{./pictures/Worst.jpg}
			\caption{Worst Case for all the Sorting Algorithms.}
		\end{figure}
		\clearpage
		\begin{figure}[h]
			\centering
			\includegraphics[width=0.75\linewidth]{./pictures/Worst_enlarged.jpg}
			\caption{Worst Case for all the Sorting Algorithms. Expanded Version}
		\end{figure}
		\item \textbf{Graph the average case running time as a function of input size $n$ for the six sorts.}
		\begin{figure}[h]
			\centering
			\includegraphics[width=0.65\linewidth]{./pictures/Average.jpg}
			\caption{Average Case for all the Sorting Algorithms.}
		\end{figure}
		\clearpage
		\begin{figure}[h]
			\centering
			\includegraphics[width=0.65\linewidth]{./pictures/Average_enlarged.jpg}
			\caption{Average Case for all the Sorting Algorithms.Expanded Version}
		\end{figure}
	\end{enumerate}
	
	\section{Analyze your data to see if the number of comparisons is correlated with execution time plot (time \ \#comparisons) vs. n and refer to these plots in your answer}
	\begin{itemize}
		\item \textbf{Quicksort:} {Showing correlation value nearly 0.99,0.99,0.99 in best and worst case but in average case it is showing 0.99,0.32,0.99 for all 3 types of quicksort variations and for all Algorithms combining it is the fastest algorithm }
		\item \textbf{Merge and Heap sort:} {Merge sort and heap sort are showing same trend in all the cases}
		\item \textbf{Insertion Sort \& Bubble sort:} { A stronger correlation with 
			n but with higher execution times compared to Merge and Heap Sort}
		\item \textbf{Radix Sort:} {Shown a linear trend bcox all the number are nearly of same digits}
	\end{itemize}
	
	
	% Add more sections as needed
	
	\vspace{1cm}
	
	\noindent \textbf{Code: }\textbf{\href{https://github.com/gdillu/DSA-Assignments/tree/main/1st-Assignments/}{Acces to Github repo}}
	
\end{document}
